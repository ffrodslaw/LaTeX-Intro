Let's create a new document to learn how to include math in \LaTeX{}. Create the necessary header so that it compiles and include this in the preamble \verb|\usepackage{amsmath, amsfonts, amsthm}|. Now you can follow along and compile as we go.

\subsection{Basic math}
 
The most basic way to type math is in-line math. We simply use the dollar sign to enter math mode. Note that either \verb|$ $| or \verb|\( \)| can be used to enter math mode. For example, \verb|$2+2=4$| produces $2+2=4$.

There are a lot of special operators that you can only use in math mode. For example, we can write \verb|$x = \log(\exp(x))$| to get $x = \log(\exp(x))$. Below is a table of very common math commands.

\begin{table}[h]
	\centering
	\caption{Useful operators}
	\label{tab:operators}
\begin{footnotesize}
    \begin{tabular}{l l}
      \hline
      Command     & Outcome \\ \hline
      \verb|\exp| & $\exp$\\
\verb|\log| & $\log$ \\
\verb|\ln| & $ln$ \\
\verb|\sqrt{x}| & $\sqrt{x}$ \\
\verb|\sin| & $\sin$ \\\
\verb|\tan| & $\tan$ \\
\verb|\cos| & $\cos$\\
\verb|\sum| & $\sum$\\
\verb|\prod| & $\prod$\\
\verb|\min| & $\min$\\
\verb|\max| & $\max$\\
\verb|\int| & $\int$ \\
\verb|\partial| & $\partial$ \\
\verb|\lim| & $\lim$ \\
\verb|\vee| & $\vee$ \\
\verb|\wedge| & $\wedge$ \\
\verb|\cup| & $\cup$ \\
\verb|\cap| & $\cap$ \\
\verb|\to| & $\to$ \\
\verb|\cdot| & $\cdot$ \\
\verb|\times| & $\times$ \\
\verb|\implies| & $\implies$ \\
\verb|\in| & $\in$ \\
\verb|\setminus| & $\setminus$ \\
\verb|\geq| & $\geq$ \\
\verb|\leq| & $\leq$ \\
\verb|\neq| & $\neq$ \\
\verb|\infty| & $\infty$ \\
\verb|\emptyset| & $\emptyset$ \\
\hline
    \end{tabular}
\end{footnotesize} 
\end{table}

We can also declare new operators in our preamble by using the function \verb|\DeclareMathOperator{\var}{var}|. This gives us a new command \verb|\var| that produces a plain text version of the variance operator. Compare

\begin{align*}
\var(x) &= \sigma^2 \\
var(x) &= \sigma^2
\end{align*}

where the top one is our new operator. We can also produce plain text words in math mode by using \verb|\text{text}|.

Writing all math in-line can make your text cramped and hard to read, so you likely want to mix it up with display math. You can use either \verb|$$ $$| or \verb|\[ \]| to enter display math. Switching to display you go from $f(x) = \exp(x)$ to $$f(x) = \exp(x)$$

In a lot of cases you will want to do this to style, ease of reading, and to emphasize important steps or results.

To add subscripts and superscripts to variables we use the underscore (\verb|_|) and the carrot (\verb|^|). For example, if we want to sum $x_i^2$ from $i=1$ to $N$, we write \verb|\sum_{i=1}^N x_i^2|. Note that we use subscript and superscript on the summation operator to set the bounds on the operation. The inline output is $\sum_{i=1}^N x_i^2$ and the display output is $$\sum_{i=1}^N x_i^2\ $$

Notice that with the bounds on the operator, there is a cosmetic difference between the display and the inline presentations. The other thing to note is the use of \verb|{i=1}| to tell the software that the whole segment is the subscript. Omitting the braces is a common mistake with obvious results
 
$$\sum_i=1^N x_i^2$$ 

Fractions are easy to implement. We simply use \verb|\frac{num}{den}| to have the top and bottom parts. Inline this gives us $\frac{x}{y}$ and display $$\frac{x}{y}\ .$$

While we're looking at fractions, let's consider the following issue.

\begin{lstlisting}
$$ \frac{1}{N} (\sum_{i=1}^N \log(f(\theta|y_i))) + (1-\frac{1}{p})X $$
\end{lstlisting}
which produces
$$\frac{1}{N} (\sum_{i=1}^N \log(f(\theta|y_i))) + (1-\frac{1}{p})X\ .$$

Look at those terrible parentheses. Unfortunately, \LaTeX{} doesn't adjust the height of them automatically, so we have to do it ourselves:

\begin{lstlisting}
$$ \frac{1}{N} \left(\sum_{i=1}^N \log(f(\theta|y_i))\right) + \left(1-\frac{1}{p}\right)X $$
\end{lstlisting}
which gives us
$$\frac{1}{N} \left(\sum_{i=1}^N \log(f(\theta|y_i))\right) + \left(1-\frac{1}{p}\right)X\ .$$

This also works for brackets and curly braces. However, because braces are special they are written with \verb|\left\{  \right\}|.

Binomials (the choose function) works the same way as fractions. Just replace \texttt{frac} with \texttt{binom} to get
$$\binom{1}{4}\ .$$

Greek letters are also important. The command for them simply is a backslash and the name of the letter. This means that $\theta$ is \verb|\theta| and capitals like $\Theta$ is with a capital letter, \verb|\Theta|.


\subsection{Fancy equations}

You might want to use numbered equations to draw attention to important results. To get equations with numbers we can use the \texttt{equation} environment.

\begin{lstlisting}
\begin{equation}\label{eq:exp}
F(t) = 1 - \exp(-\lambda t)
\end{equation}
\end{lstlisting}

which produces 
\begin{equation}\label{eq:exp}
F(t) = 1 - \exp(-\lambda t)\ .
\end{equation}

Note that the use of the label command works exactly like it did with figures and tables so we can refer to Equation \ref{eq:exp} by using \verb|\ref{eq:exp}|.

Being able to line up multiple equations creates a nice clean visual effect. The \texttt{align} environment provides a system for doing that. The code

\begin{lstlisting}
\begin{align*}
F(t) &= 1- \exp(-\lambda t)\\
F(0) &= 1-\exp(0)\\
     &= 0.
\end{align*}
\end{lstlisting}

Notice that this combines a few things we have done so far. The \* removes numbers, just like it did with sections, while the \& is what lines equations up and the \textbackslash\textbackslash breaks the line, just like with tables. If we remove the \* we get equation numbers for each line that can be assigned labels.

Piecewise functions are also important. The \texttt{case} environment inside math mode helps us here.

\begin{lstlisting}
$$
f(t) =
\begin{cases}
	1 - \exp(-\lambda t) & \text{if } t >0 \\
	0                    & \text{otherwise.}
\end{cases}
$$
\end{lstlisting}

The code produces

$$
f(t) =
\begin{cases}
	1 - \exp(-\lambda t) & \text{if } t >0 \\
	0                    & \text{otherwise.}
\end{cases}
$$

Notice the use of \verb|\text{}| to produce normal-looking text inside math. Also notice that I included the space after ``if'' because math mode otherwise runs things together without spaces.

\subsection{Matrices}

At some point you may find yourself with the inevitable task of writing a matrix. Here's how you would do that.

\begin{lstlisting}
$$
A =
\begin{bmatrix*}
	a_{11} & a_{12} \\
	a_{21} & a_{22} \\
\end{bmatrix*}
$$
\end{lstlisting}

And this gives you 
$$
A =
\begin{bmatrix*}
a_{11} & a_{12} \\
a_{21} & a_{22} \\
\end{bmatrix*}
$$

The \texttt{b} in \texttt{bmatrix} refers to the brackets. You can also use parentheses. For instance

$$
A =
\begin{pmatrix}
	a_{1,1} & a_{1,2} & \ldots & a_{1,n} \\
	a_{2,1} & a_{2,2} & \ldots & a_{2,n} \\
	\vdots & \vdots   & \ddots & \vdots \\
	a_{m,1} & a_{m,2} & \ldots & a_{m,n}  \\
\end{pmatrix}
$$

is made by

\begin{lstlisting}
$$
A =
\begin{pmatrix}
	a_{1,1}	& a_{1,2} & \ldots & a_{1,n} \\
	a_{2,1} & a_{2,2} & \ldots & a_{2,n} \\
	\vdots 	& \vdots  & \ddots & \vdots \\
	a_{m,1} & a_{m,2} & \ldots & a_{m,n} 
\end{pmatrix}
$$
\end{lstlisting}

\subsection{Theorems and proofs}

The \texttt{theorem} and \texttt{proof} environments are the basis for most formal theoretical analysis. You setup these enviroments by declaring theorems in your preamble. 

\begin{lstlisting}
\newtheorem{thm}{Theorem}
\newtheorem{claim}{Claim}
\newtheorem{proposition}{Proposition}
\newtheorem{lemma}{Lemma}
\end{lstlisting}

The code defines four environments that we can use. Let's go ahead and write a theorem with the following.

\begin{lstlisting}
\begin{thm}\label{th:subsets}
If $A$ is a subset of $B$ and $B$ is a subset of $C$ then $A$ is a subset of $C$.
\end{thm}
\end{lstlisting}

\begin{thm}\label{th:subsets}
If $A$ is a subset of $B$ and $B$ is a subset of $C$ then $A$ is a subset
of $C$.
\end{thm}

Now we have the statement of Theorem \ref{th:subsets}. For the proof we use the \texttt{proof} environment.

\begin{lstlisting}
\begin{proof}
Left as an exercise to the reader.
\end{proof}
\end{lstlisting}

\begin{proof}
Left as an exercise to the reader.
\end{proof}

\subsection{Fancy fonts}

Table \ref{tab:fonts} shows some additional things you may want to use in math mode.

\begin{table}[h]
	\centering
	\label{tab:fonts}
	\caption{Fonts in math mode}
\begin{small}
  \begin{center}
    \begin{tabular}{c c c}
      \hline
Command & Use & Example \\ \hline
\verb|\mathbb{}| & Blackboard & $\mathbb{R}$ is used to denote real numbers.\\
\verb|\mathbf{}| & Boldface & $\mathbf{X}$ bold is sometimes used for matrices.\\
\verb|\mathfrak{}| & Fraktur & $\mathfrak{R}$ is also sometimes used to denote real numbers.\\
\verb|\hat{}| & Add a hat to a variable & $\hat{\theta}$ is an estimate of $\theta$.\\
\verb|\tilde{}| & Add a tilde & $\tilde{\theta}$ is sometimes used to denote a specific value of $\theta$.\\
\verb|\bar{}| & Add a bar & $\bar{x}$ is often used to denote the mean of $x$.\\



      \hline
    \end{tabular}
  \end{center}
\end{small}
\end{table}



%\subsection{Exercise} 